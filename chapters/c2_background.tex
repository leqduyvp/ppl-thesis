\chapter{Kiến thức nền tảng và tổng quan nghiên cứu}

\section{Kiến thức nền tảng}
Du lịch văn hóa là một trong những lĩnh vực quan trọng của ngành du lịch, nơi mà giá trị cốt lõi nằm ở việc khai thác, bảo tồn và phát huy các di sản văn hóa vật thể và phi vật thể. Các hoạt động du lịch văn hóa không chỉ đơn thuần mang tính giải trí mà còn góp phần giáo dục, lan tỏa tri thức và quảng bá hình ảnh đất nước. Đối với Việt Nam, với nguồn tài nguyên văn hóa phong phú và đa dạng, việc khai thác hiệu quả các giá trị này có ý nghĩa to lớn đối với sự phát triển bền vững của ngành du lịch.  
Trong xu thế phát triển của khoa học và công nghệ, đặc biệt là công nghệ thông tin, các hình thức số hóa di sản ngày càng được quan tâm. Công nghệ số cho phép các giá trị văn hóa được lưu trữ, tái hiện và phổ biến rộng rãi thông qua các nền tảng trực tuyến, ứng dụng di động hay các hệ thống tương tác đa phương tiện. Việc số hóa không chỉ giúp nâng cao trải nghiệm của du khách mà còn tạo điều kiện thuận lợi cho công tác bảo tồn và truyền bá di sản đến đông đảo cộng đồng.  
Một trong những công nghệ nền tảng có vai trò then chốt trong việc hỗ trợ du lịch văn hóa là công nghệ bản đồ số kết hợp với hệ thống định vị toàn cầu (GPS). Các dịch vụ dựa trên vị trí (Location Based Services – LBS) cho phép xác định vị trí người dùng, từ đó cung cấp thông tin định hướng và gợi ý nội dung phù hợp. Điều này mở ra khả năng xây dựng những ứng dụng giúp du khách dễ dàng định vị, tìm đường đi và tiếp cận thông tin về các điểm di sản trong hành trình tham quan.  
Những kiến thức nền tảng này là cơ sở quan trọng để khóa luận tập trung vào việc xây dựng một ứng dụng bản đồ số hỗ trợ trải nghiệm di sản văn hóa. Trên nền tảng các công nghệ định vị, bản đồ số và tương tác đa phương tiện, khóa luận kế thừa những tiến bộ sẵn có và phát triển giải pháp phù hợp với bối cảnh thực tế tại Việt Nam.  


\section{Tổng quan các nghiên cứu và hệ thống liên quan}
\subsection{Ứng dụng trong và ngoài nước về số hóa di sản}
\subsection{Các nền tảng hỗ trợ trải nghiệm di sản dựa trên bản đồ}
\subsection{Ưu điểm và hạn chế của các giải pháp hiện có}

\section{Đánh giá và định hướng}
\subsection{Khoảng trống nghiên cứu}
\subsection{Định hướng phát triển ứng dụng TMAP}
