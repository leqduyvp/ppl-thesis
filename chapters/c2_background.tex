\chapter{Các kiến thức liên quan}

\section{Tổng quan về trải nghiệm du lịch văn hóa và lịch sử tại Việt Nam}
Du lịch văn hóa là một trong những hình thức du lịch đã xuất hiện từ lâu, không
chỉ ở riêng Việt Nam mà còn ở trên thế giới. Nó vẫn luôn là một trong những loại
hình phát triển bậc nhất, giữa vô vàn các loại hình mới như du lịch nghỉ dưỡng, du
lịch sinh thái, du lịch mạo hiểm, …

Du lịch văn hóa là sự di chuyển của con người đến các điểm du lịch văn hóa ở
các quốc gia hay vùng miền không phải nơi họ sống, với mục đích khám phá, mở
rộng kiến thức, kinh nghiệm về nhu cầu văn hóa của họ. Tồn tại ở dạng một loại hình
du lịch của ngành du lịch, và cũng là một ngành kinh doanh có sử dụng các yếu tố
văn hóa như phong tục tập quán, tín ngưỡng, các lễ hội truyền thống, di tích lịch sử,
những kiến trúc, nghệ thuật và các sản phẩm văn hóa khác. Nói cách khác, du lịch
văn hóa cũng là sự khai thác các tài nguyên du lịch văn hóa bao gồm những di tích
lịch sử - văn hóa, công trình kiến trúc nghệ thuật, di tích cách mạng, các giá trị văn
hóa dân gian, lễ hội truyền thống, các công trình lao động sáng tạo của con người,
hoặc khai thác các yếu tố văn hóa tâm linh làm cơ sở và mục tiêu đáp ứng nhu cầu
tâm linh của con người.

Việt Nam ta có lượng tài nguyên du lịch văn hóa dồi dào, thể hiện tiềm năng
phát triển vượt bậc của du lịch văn hóa. Hơn 4000 năm dựng nước và giữ nước, 54
dân tộc anh em đã tạo ra các giá trị truyền thống văn hóa đồ sộ. Bên cạnh đó không
thể không kể đến lượng lớn hơn 44000 danh lam thắng cảnh, di tích lịch sử - với hơn
3000 địa danh là di sản cấp quốc gia, hơn 5000 địa danh cấp tỉnh và 7 di sản văn hóa
được UNESCO công nhận là di sản văn hóa thế giới, cùng 117 bảo tàng và 8000 lễ
hội \cite{huong2022}.

Những con số kể trên phần nào cũng đã thể hiện được ý nghĩa và tầm quan trọng
của du lịch văn hóa. Không chỉ mang về nguồn thu ngân sách khổng lồ với vai trò là
một ngành kinh doanh, nó cũng mang lại những giá trị to lớn về mặt duy trì và bảo
tồn văn hóa.

Về phân loại, ta có thể chia nhỏ các hình thức du lịch văn hóa làm 3 loại hình
nhỏ \cite{vinpearl2022}: