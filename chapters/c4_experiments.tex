\begin{center}
\textbf{\large{Kết luận}	}
\end{center}
\addcontentsline{toc}{chapter}{Kết luận}

\textbf{Kết quả đạt được}

Khóa luận đã hoàn thành phát triển hệ thống TMAP với
hai phần là trình biên soạn trên Website và ứng dụng trải nghiệm trên điện thoại. Phần
trình biên soạn đã được triển khai và có thể được sử dụng trên các trình duyệt phổ
thông, đáp ứng được nhu cầu tạo trải nghiệm của các đơn vị văn hóa. Phần ứng dụng
trải nghiệm đã được đóng gói và có thể sẵn sàng phát hành cho đông đảo người sử
dụng. TMAP qua đó đã đáp ứng được những đầu mục cơ bản của bản đồ số trong du
lịch văn hóa, hỗ trợ được phần nào trải nghiệm của du khách trong quá trình khám
phá các địa điểm du lịch.

Ứng dụng cũng đóng góp một phần nhỏ trong quá trình hoàn thiện dự án lớn
Trealet, tham gia vào xu hướng ứng dụng công nghệ thông tin để quảng bá và giáo
dục văn hóa du lịch đang ngày càng lớn mạnh. Kết quả của khóa luận sẽ được ứng
dụng tại Bảo tàng Văn hóa các Dân tộc Việt Nam ở Thái Nguyên.

\textbf{Hướng phát triển}

Các hình thức tương tác và mô tả điểm du lịch của TMAP
có nhiều tiềm năng để phát triển đa dạng hơn, bên cạnh đó phần giao diện vẫn còn
đơn giản sẽ là những ưu tiên đầu trong việc sửa đổi và phát triển TMAP. Hi vọng có thể biến TMAP thành sản
phẩm có thể được sử dụng rộng rãi trong truyền bá, giáo dục các địa điểm văn hóa.