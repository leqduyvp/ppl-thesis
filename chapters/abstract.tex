\begin{center}
\textbf{\large{Tóm tắt}	}
\end{center}
\addcontentsline{toc}{chapter}{Tóm tắt}

\textbf{Tóm tắt:} Việt Nam có nguồn tài nguyên du lịch phong phú với sự đa dạng về địa lý, cảnh quan và truyền thống văn hóa của 54 dân tộc. Trong đó, du lịch văn hóa là một phân nhánh quan trọng, vừa góp phần quảng bá hình ảnh đất nước tới bạn bè quốc tế, vừa tạo động lực cho thế hệ trẻ khám phá, bảo tồn và phát triển các giá trị văn hóa dân tộc. Tuy nhiên, trải nghiệm của du khách tại các điểm di sản vẫn còn nhiều hạn chế do thiếu công cụ hỗ trợ tương tác và hướng dẫn trực quan.

Luận văn này đề xuất và phát triển ứng dụng \textbf{TMAP}, được tích hợp trong nền tảng \textbf{Trealet} của phòng nghiên cứu HMI, nhằm nâng cao trải nghiệm của du khách khi tham quan các di sản văn hóa. Ứng dụng bao gồm hai thành phần chính: (i) \textit{trình biên tập} dành cho các đơn vị văn hóa hoặc cá nhân, cho phép tạo bản đồ số các điểm tham quan kèm mô tả dưới nhiều hình thức (văn bản, hình ảnh, video, âm thanh, câu hỏi tương tác), và (ii) \textit{trình trải nghiệm} dành cho du khách, hỗ trợ định vị trên bản đồ, theo dõi hành trình và tham gia các hoạt động tương tác số.

Kết quả cho thấy ứng dụng có khả năng hỗ trợ các đơn vị văn hóa xây dựng nội dung số một cách linh hoạt, đồng thời mang lại trải nghiệm trực quan và tương tác cao cho du khách. Đây là bước thử nghiệm ban đầu hướng tới việc số hóa và phổ biến di sản văn hóa Việt Nam trên nền tảng công nghệ thông tin.

\textbf{Từ khóa:} TMAP, Trealet, du lịch văn hóa, di sản văn hóa, ứng dụng di động
