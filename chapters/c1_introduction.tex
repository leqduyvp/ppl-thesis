\chapter{Giới thiệu}

\section{Đặt vấn đề}
Trong bối cảnh cách mạng công nghiệp 4.0 diễn ra mạnh mẽ, công nghệ thông tin ngày càng chứng tỏ vai trò quan trọng trong nhiều lĩnh vực của đời sống xã hội, trong đó có văn hóa và du lịch. Việc ứng dụng công nghệ số không chỉ nâng cao hiệu quả trong công tác quản lý, bảo tồn di sản, mà còn tạo ra các hình thức trải nghiệm mới mẻ và hấp dẫn cho du khách. 

Việt Nam là quốc gia giàu tiềm năng với hàng nghìn di sản văn hóa vật thể và phi vật thể được công nhận ở cấp quốc gia và quốc tế. Tuy nhiên, hoạt động quảng bá và khai thác giá trị di sản hiện nay vẫn còn nhiều hạn chế. Thông tin tại các điểm tham quan thường dừng ở mức cơ bản, thiếu tính trực quan và tương tác. Điều này khiến du khách, đặc biệt là giới trẻ, chưa có nhiều hứng thú trong việc tiếp nhận tri thức văn hóa. Chính vì vậy, nhu cầu về một công cụ hỗ trợ số hóa, trực quan hóa và nâng cao trải nghiệm du khách tại các di sản ngày càng trở nên cấp thiết.  

\section{Mục tiêu nghiên cứu}
Khóa luận hướng tới việc xây dựng một ứng dụng có khả năng hỗ trợ du khách trong quá trình tham quan và khám phá di sản văn hóa. Cụ thể, ứng dụng cần cung cấp cho các đơn vị văn hóa hoặc cá nhân một công cụ biên tập, cho phép chủ động xây dựng bản đồ du lịch số, tích hợp nội dung đa phương tiện và thiết kế hoạt động tương tác. Đồng thời, ứng dụng phải đáp ứng nhu cầu trải nghiệm của du khách với các tính năng định vị, định hướng và tương tác trực tuyến.  

\section{Đối tượng và phạm vi nghiên cứu}
Đối tượng nghiên cứu chính của khóa luận là các giải pháp và công nghệ hỗ trợ số hóa di sản văn hóa, đặc biệt trong lĩnh vực ứng dụng bản đồ số và trải nghiệm di sản thông qua thiết bị di động. Phạm vi nghiên cứu tập trung vào việc phát triển ứng dụng \textbf{TMAP} trên nền tảng \textbf{Trealet} do phòng nghiên cứu HMI phát triển, và thử nghiệm trong một số tình huống tham quan di sản cụ thể.  

\section{Phương pháp nghiên cứu}
Khóa luận xây dựng cơ sở lý thuyết dựa trên việc khảo sát các địa điểm du lịch có thể áp dụng tại Hà Nội, kèm theo các ràng buộc và tính năng của nền tảng Trealet. Tiếp đó, dựa trên cơ sở lý thuyết, thiết kế và xây dựng ứng dụng TMAP tích hợp vào nền tảng. Cuối cùng, tiến hành triển khai thử nghiệm và đánh giá kết quả để rút ra nhận xét và đề xuất hướng phát triển tiếp theo.  

\section{Ý nghĩa và đóng góp}
Khóa luận đóng góp một giải pháp ứng dụng CNTT trong lĩnh vực văn hóa và du lịch. Ứng dụng TMAP không chỉ giúp các đơn vị văn hóa số hóa và biên tập nội dung tham quan một cách linh hoạt, mà còn mang lại cho du khách trải nghiệm trực quan, sinh động và có tính tương tác cao. Kết quả của khóa luận là bước thử nghiệm quan trọng chứng minh khả năng ứng dụng công nghệ thông tin vào bảo tồn và phát huy giá trị văn hóa, đồng thời mở ra hướng phát triển các công cụ số trong du lịch văn hóa.  

\section{Cấu trúc khóa luận}

Trên đây là chương 1 bao gồm phần mở đầu, trình bày khái quát bối cảnh, mục tiêu, phạm vi, phương pháp nghiên cứu cũng như ý nghĩa và đóng góp chính. Phần còn lại của khóa luận bao gồm 4 chương: Chương 2 trình bày kiến thức nền tảng liên quan. Chương 3 mô tả chi tiết thiết kế hệ thống và phương pháp xây dựng ứng dụng TMAP. Chương 4 trình bày các thực nghiệm, công cụ triển khai và kết quả đánh giá. Cuối cùng, Chương 5 đưa ra kết luận, nêu những hạn chế còn tồn tại và đề xuất các hướng phát triển trong tương lai.
