\chapter{Giới thiệu}

\setcounter{page}{1}
\pagenumbering{arabic}
\section{Đặt vấn đề}

Với nguồn tài nguyên văn hóa hình thành từ hàng ngàn năm lịch sử, sự đa dạng do 54 dân tộc cùng các phong tục tập quán và tín ngưỡng tạo nên, kết hợp cùng hệ thống danh lam thắng cảnh và di tích lịch sử, du lịch văn hóa tại Việt Nam được xem là một trong những loại hình quan trọng của ngành du lịch. Loại hình này vừa đóng góp giá trị kinh tế, vừa góp phần bảo tồn, phát triển văn hóa, đồng thời quảng bá kho tàng văn hóa dân tộc ra thế giới.  

Sự phát triển công nghệ hiện nay đóng vai trò quan trọng trong việc số hóa và quảng bá di sản văn hóa. Trong bối cảnh Cách mạng công nghiệp lần thứ tư, các công nghệ như trí tuệ nhân tạo, điện toán đám mây, Internet vạn vật, công nghệ 3D, thực tế ảo đã được ứng dụng để cải thiện trải nghiệm du lịch. Các sản phẩm du lịch số và di sản số ngày càng gia tăng cả về số lượng và chất lượng, mở ra hướng tiếp cận mới cho du khách, đồng thời khắc phục hạn chế của hình thức truyền thống.  

Một trong những hạn chế lớn tại các điểm di sản và di tích lịch sử là nhu cầu định hướng thông tin chính xác và có hệ thống. Trong mô hình truyền thống, yêu cầu này thường được đáp ứng nhờ hướng dẫn viên thông qua các dịch vụ lữ hành, dẫn đến khó khăn đối với các nhóm khách nhỏ do chi phí và sự khác biệt chất lượng dịch vụ.  

Việc tự tìm hiểu thông tin trên Internet tuy khả thi nhưng dữ liệu thường rời rạc, thiếu hệ thống định hướng, thậm chí có thể sai lệch về địa chỉ hoặc nội dung. Điều này ảnh hưởng trực tiếp đến trải nghiệm độc lập của du khách. Ngoài ra, các đơn vị quản lý và bảo tồn văn hóa cũng cần công cụ hỗ trợ để xây dựng và tối ưu hóa trải nghiệm số, vừa thu hút khách tham quan vừa phục vụ mục tiêu giáo dục.  

\section{Bài toán}

Đáp ứng nhu cầu trên, phòng Thí nghiệm Tương tác Người – Máy (Khoa Công nghệ Thông tin, Trường Đại học Công nghệ – Đại học Quốc gia Hà Nội) đã phát triển dự án \textbf{Trealet} nhằm xây dựng nền tảng hỗ trợ số hóa trải nghiệm tại các điểm du lịch văn hóa.  

\textbf{Bài toán chung của Trealet}  

Các đơn vị văn hóa sở hữu nguồn dữ liệu lớn về di sản nhưng gặp khó khăn trong đồng bộ lưu trữ và trình bày. Du khách cần trải nghiệm mới, trực quan, dễ tiếp cận qua môi trường số và các thiết bị kết nối Internet. Hệ thống Trealet đã bước đầu tích hợp nhiều công nghệ hiện đại để trình bày nội dung văn hóa, hỗ trợ tham quan bảo tàng dưới dạng trải nghiệm số, trò chơi dân gian trên nền tảng thực tế tăng cường, v.v. Một số đơn vị đã thử nghiệm và cung cấp phản hồi cho quá trình hoàn thiện.  

\textbf{Bài toán riêng của khóa luận – phát triển ứng dụng TMAP}  

Khóa luận tập trung phát triển hệ thống con TMAP, nhằm cung cấp trải nghiệm du lịch có định hướng, đảm bảo tính chính xác của thông tin và vị trí điểm du lịch, đồng thời bổ sung khả năng tương tác.  

Hệ thống yêu cầu:  
\begin{itemize}
    \item Một trình biên tập trải nghiệm bản đồ dành cho các đơn vị, cho phép chọn vị trí thực tế trên bản đồ, sắp xếp thứ tự điểm tham quan.  
    \item Khả năng biên soạn nội dung mô tả, tải lên dữ liệu đa phương tiện (ảnh, video, âm thanh), bổ sung các yếu tố tương tác.  
    \item Giao diện trải nghiệm dành cho du khách với chức năng định vị, hiển thị thông tin, ghi nhận tương tác (ảnh, âm thanh, nhận xét, trả lời trắc nghiệm) và theo dõi tiến trình tham quan.  
    \item Cơ chế bảo mật thông tin người dùng thông qua tài khoản đăng nhập.  
    \item Khả năng triển khai trên nền tảng di động để đáp ứng tính linh hoạt của trải nghiệm bản đồ.  
\end{itemize}

\section{Phương pháp đề xuất}

Khóa luận đề xuất phát triển ứng dụng \textbf{TMAP} và tích hợp vào nền tảng Trealet, bao gồm hai thành phần:  
\begin{itemize}
    \item \textbf{Trình biên soạn}: được phát triển trên nền tảng Web, hỗ trợ các đơn vị và nhà sáng tạo xây dựng bản đồ trải nghiệm với danh sách điểm tham quan, thông tin mô tả (văn bản và nội dung đa phương tiện), cùng các chức năng tương tác.  
    \item \textbf{Trình trải nghiệm}: được phát triển trên nền tảng di động, cung cấp giao diện đăng ký/đăng nhập, cho phép du khách chọn và tham gia các bản đồ trải nghiệm. Các bản đồ được hiển thị trên Google Map, du khách có thể tương tác trực tiếp với từng điểm tham quan thông qua thông tin và chức năng đã được thiết kế trong phần biên tập.  
\end{itemize}

\section{Cấu trúc của khóa luận}
Sau phần giới thiệu của chương 1, phần còn lại của khóa luận sẽ có nội dung
được cấu trúc như sau:
Chương 2 - Giới thiệu về trải nghiệm du lịch văn hóa tại Việt Nam, các công
nghệ được ứng dụng để phát triển hệ thống, cũng như các công nghệ liên quan,
Chương 3 - Tổng quan về Trealet và TMAP – Giới thiệu tổng quan về hệ thống
Trealet và TMAP, cũng như chi tiết hơn về các phần thiết kế của TMAP – như luồng
dữ liệu, các thiết kế về cơ sở dữ liệu, chức năng giao diện, liên kết hệ thống tổng
quan.
Cuối cùng là phần Kết luận – Tóm tắt về các thành quả và hạn chế nói chung
của TMAP trong khóa luận, đề xuất các hướng phát triển và duy trì về sau.